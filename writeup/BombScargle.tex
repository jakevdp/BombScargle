\documentclass[12pt,pdftex]{article}

\begin{document}

\section{Lomb-Scargle Periodogram}
The Lomb-Scargle Periodogram is fundamentally a measure of a normalized chi-squared for fitting a sinusoidal model to data. Given data $\{t_j, y_j\}$ with uniform errors $\sigma$ we can choose a single-term linear sinusoidal model defined by the frequency $\omega$ and amplitudes $a$ and $b$:

\begin{equation}
  f(t|\omega, a, b) \equiv a\sin(\omega t) + b\cos(\omega t)
\end{equation}

Given this model, we can write the likelihood

\begin{equation}
  L \equiv p(\{t_j, y_j\}, \sigma~|~\omega, a, b) =
  \prod_{j=1}^{N} \frac{1}{\sqrt{2\pi\sigma^2}} \exp\left(
  \frac{-[y_j - f(t_j|a, b, \omega)]^2}{2\sigma^2}
  \right)
\end{equation}

For any choice of $\omega$, we can find values $[a_0(\omega), b_0(\omega)]$ which maximize this likelihood. The goodness-of-fit can be determined by evaluating $\chi^2$ at this maximum:

\begin{equation}
  \chi^2(\omega) = \frac{1}{\sigma^2}\sum_{j=1}^N[y_j - f(t|\omega, a_0(\omega), b_0(\omega))]^2
\end{equation}

If we define $\chi_0^2 = \sigma^{-2}\sum y_j^2$, then we can write the *Lomb-Scargle Periodogram* as

\begin{equation}
  P_{LS}(\omega) \equiv 1 - \frac{\chi^2(\omega)}{\chi_0^2}
\end{equation}

This periodogram is a normalized measure of the goodness-of-fit of a sinusoidal model with frequency $\omega$. Here I've purposely left-out the actual computation of $a_0(\omega)$ and $b_0(\omega)$, as well as the explicit expression to compute $P_{LS}$; we'll get to that below.

\section{Computing $P_{LS}$}
We can compute $P_{LS}$ using the above formalism.
For later convenience, let's write our model in the form of a matrix-vector product:

\begin{equation}
  f(t|\omega, \theta) = X_\omega \theta
\end{equation}

Where in the simple case above, $\theta = [a, b]^T$ and

\begin{equation}
  X_\omega = \left[\begin{array}{lll}
    \sin\omega t_1 && \cos\omega t_1\\
    \sin\omega t_2 && \cos\omega t_2\\
     & \vdots &\\
    \sin\omega t_N && \cos\omega t_N
  \end{array}\right]
\end{equation}

With this form, the $\chi^2$ can be written

\begin{equation}
\chi^2(\omega, \theta) = \frac{1}{\sigma^2}(y - X_\omega b)^T(y - X_\omega b)
\end{equation}

This can be minimized by the standard means to find the best-fit $b$:

\begin{equation}
  b_0 = (X_\omega^TX_\omega)^{-1}X_\omega^Ty
\end{equation}

Plugging this back in to the expression for $\chi^2$ gives

\begin{equation}
  \chi^2(\omega) = \frac{1}{\sigma^2}\left[
    y^Ty - y^TX_\omega(X_\omega^TX_\omega)^{-1}X_\omega^Ty
    \right]
\end{equation}

and we see that

\begin{equation}
  P_{LS}(\omega) = \frac{y^TX_\omega(X_\omega^TX_\omega)^{-1}X_\omega^Ty}{y^Ty}
\end{equation}

If we expand the above matrix expression, we can re-express this equation in terms of sums of sines and cosines of the input data, but that can be found elsewhere. One important piece of the computation is the ability to quickly compute the above expression over a large number of frequencies using the FFT, but we won't get into that here.

 The main point here is that fundamentally, the Lomb-Scargle periodogram is a normalized measure of the $\chi^2$ for a maximum-likelihood model fit. Further, by changing the definition of $X_\omega$ and adding more columns, we can quite easily account for an arbitrary offset $\mu$, include non-uniform errors $\sigma_j$, compute the periodogram for an arbitrary multi-frequency model, etc.

\section{Making this Robust}
A well-known problem with the Lomb-Scargle periodogram is its lack of robustness to outliers: the Gaussian form of the likelihood expression means that if the errors $\sigma_j$ are mis-specified, the outlying point(s) will have an extremely large effect on the final fit. What is required is to replace the above $\chi^2$ computation with a robust model that can account for these errors.

\end{document}
